\mfsloadaseql{c1}{comm}{%
Source text sentence number 1.
Source text sentence number 2.
Source text sentence number 3.
comm A A A A A A A A A A A A A A A A A A A A A A A A A A A A A A A A A A A A A A A A A A A A A A A A A A A A A A A A A A A A A A A A A A A A A A A A A A A A A A A A A A A A A A A A A A A A A A A A A A A A A A A A A A A A A A A A A A A A A A A A A A A A A A A A A A A A A A 

A A A A A A A A A A A A A A A A A A A A A A A A A A A A A A A A A A A A A A A A A A A A A A A A A A A A A A A A A A A A A A A A A A A A A A A A A A A A A A A A A A A A A A A A A A A A A A A A A A A A A A A A A A A A A A A A A A A A A A A A A A A A A A A A A A A A A A 
comm Text B Text B Text B Text B Text B Text B Text B Text B Text B Text B Text B Text B Text B Text B Text B Text B Text B Text B Text B Text B Text B Text B Text B Text B Text B Text B Text B Text B Text B Text B Text B Text B Text B Text B Text B Text B Text B Text B Text B Text B Text B Text B Text B Text B Text B Text B Text B Text B Text B Text B Text B Text B Text B Text B Text B Text B Text B Text B Text B Text B 

Text B Text B Text B Text B Text B Text B Text B Text B Text B Text B Text B Text B Text B Text B Text B Text B Text B Text B Text B Text B Text B Text B Text B Text B Text B Text B Text B Text B Text B Text B Text B Text B Text B Text B Text B Text B Text B Text B Text B Text B Text B Text B Text B Text B Text B Text B Text B Text B Text B Text B Text B Text B Text B Text B Text B Text B Text B Text B Text B Text B 

Text B Text B Text B Text B Text B Text B Text B Text B Text B Text B Text B Text B Text B Text B Text B Text B Text B Text B Text B Text B Text B Text B Text B Text B Text B Text B Text B Text B Text B Text B Text B Text B Text B Text B Text B Text B Text B Text B Text B Text B Text B Text B Text B Text B Text B Text B Text B Text B Text B Text B Text B Text B Text B Text B Text B Text B Text B Text B Text B Text B 
comm Text C Text C Text C Text C Text C Text C Text C Text C Text C Text C Text C Text C Text C Text C Text C Text C Text C Text C Text C Text C Text C Text C Text C Text C Text C Text C Text C Text C Text C Text C Text C Text C Text C Text C Text C Text C 
}

%==================================
\mfsloadaseql{c2}{commx}{%
Extracts from Wikipedia for the Solar System, Venus, and Mars.
commx The Solar System[c] is the gravitationally bound system of the Sun and the objects that orbit it. It formed 4.6 billion years ago from the gravitational collapse of a giant interstellar molecular cloud. The vast majority (99.86\%) of the system's mass is in the Sun, with most of the remaining mass contained in the planet Jupiter. The four inner system planets—Mercury, Venus, Earth and Mars—are terrestrial planets, being composed primarily of rock and metal. The four giant planets of the outer system are substantially larger and more massive than the terrestrials. The two largest, Jupiter and Saturn, are gas giants, being composed mainly of hydrogen and helium; the next two, Uranus and Neptune, are ice giants, being composed mostly of volatile substances with relatively high melting points compared with hydrogen and helium, such as water, ammonia, and methane. All eight planets have nearly circular orbits that lie near the plane of Earth's orbit, called the ecliptic. 
commx Venus is the second planet from the Sun. It is sometimes called Earth's "sister" or "twin" planet as it is almost as large and has a similar composition. As an interior planet to Earth, Venus (like Mercury) appears in Earth's sky never far from the Sun, either as morning star or evening star. Aside from the Sun and Moon, Venus is the brightest natural object in Earth's sky, capable of casting visible shadows on Earth at dark conditions and being visible to the naked eye in broad daylight.[18][19]

Venus is the second largest terrestrial object of the Solar System, with a surface gravity minimally lower than on Earth, but having only an induced magnetosphere. The carbon dioxide atmosphere of Venus is the densest of the four terrestrial planets. The atmospheric pressure at the planet's surface is about 92 times the sea level pressure of Earth, or roughly the pressure at 900 m (3,000 ft) underwater on Earth. Even though Mercury is closer to the Sun, Venus has the hottest surface of any planet in the Solar System, with a mean temperature of 737 K (464 °C; 867 °F). Venus is shrouded by an opaque layer of highly reflective clouds of sulfuric acid, making it the planet with the highest albedo in the Solar System and preventing its surface from being seen from Earth in light. It may have had water oceans in the past,[20][21] but after these evaporated the temperature rose under a runaway greenhouse effect.[22] The water has probably photodissociated, and the free hydrogen has been swept into interplanetary space by the solar wind because of the lack of an internally induced magnetic field.[23] At roughly 50 km above the surface atmospheric conditions reach Earth-like temperatures and levels of pressure. The possibility of life on Venus has long been a topic of speculation but convincing evidence has yet to be found. 
commx Mars is the fourth planet from the Sun and the second-smallest planet in the Solar System, being larger than only Mercury. In the English language, Mars is named for the Roman god of war. Mars is a terrestrial planet with a thin atmosphere (less than 1% that of Earth's), and has a crust primarily composed of elements similar to Earth's crust, as well as a core made of iron and nickel. Mars has surface features such as impact craters, valleys, dunes, and polar ice caps. It has two small and irregularly shaped moons: Phobos and Deimos.

Some of the most notable surface features on Mars include Olympus Mons, the largest volcano and highest known mountain on any planet in the Solar System, and Valles Marineris, one of the largest canyons in the Solar System. The Borealis basin in the Northern Hemisphere covers approximately 40\% of the planet and may be a large impact feature.[20] Days and seasons on Mars are comparable to those of Earth, as the planets have a similar rotation period and tilt of the rotational axis relative to the ecliptic plane. Liquid water on the surface of Mars cannot exist due to low atmospheric pressure, which is less than 1\% of the atmospheric pressure on Earth.[21][22] Both of Mars's polar ice caps appear to be made largely of water.[23][24] In the distant past, Mars was likely wetter, and thus possibly more suited for life. However, it is unknown whether life has ever existed on Mars.
}

%==================================
\mfsloadaseql{c3}{commx}{%
Treatise upon dandelions
commx Taraxacum (/təˈræksəkʊm/) is a large genus of flowering plants in the family Asteraceae, which consists of species commonly known as dandelions. The scientific and hobby study of the genus is known as taraxacology.[3] The genus is native to Eurasia and North America, but the two most commonplace species worldwide, \textit{T. officinale} (the common dandelion) and \textit{T. erythrospermum} (the red-seeded dandelion), were introduced from Europe into North America, where they now propagate as wildflowers.[4] Both species are edible in their entirety.[5] The common name dandelion (/ˈdændɪlaɪ.ən/ DAN-di-ly-ən, from French \textit{dent-de-lion}, meaning 'lion's tooth') is also given to specific members of the genus.
commx Taraxacum est un genre de plantes dicotylédones anémochores appartenant à la famille des Asteraceae (Composées). C'est le genre des « pissenlits » véritables.
commx Caisearbhán a thugtar ar dhá chineál planda den ghéineas \textit{Taraxacum} is den fhine Asteraceae. Caisearbhán coiteann (\textit{T. officinale}) agus caisearbhán deargshíolach (\textit{T. erythrospermum}) is ainmneacha dóibh is tá siad dúchasach don Eoráise is do Mheiriceá Thuaidh, ach tá siad flúirseach ar fud na cruinne. Tá an dá speiceas inite ina n-iomláine. Fearacht gach ball den fhine Asteraceae, tá mionbhláthanna acu cruinnithe le chéile i mbláthcheann ilchodach. Tugtar bláthóg ar gach aon bhláth sa bhláthcheann. Baineann an-chuid ball den ghéineas Taraxacum úsáid as apaimiscis chun síolta a tháirgeadh go héagnéasach, 'sé sin le rá in éagmais pailniú, rud a fhágann go bhfuil sliocht gach máthairphlanda díreach cosúil leis ó thaobh na géineolaíochta de. 
}
\mfsloadaseql{c4}{commx}{%
Test
commx A
commx B
commx C
}